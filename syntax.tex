
\section{HIJSON syntax}\label{hijson-syntax}

In Figure~\ref{fig:HIJSON} it is shown a simplified example of an input file ready to be processed by the HIJSON pipeline.

\begin{figure}[h]
\begin{verbatim}
{
  "config": {
  ...
  },
 "data": 
 [
 ...
 {
 "id": "architecture",
 "type": "FeatureCollection",
 "features": [...] 
 },
 {
 "id": "furniture_1",
 "type": "FeatureCollection",
 "features": [...] 
 },
 ...
 ]
}
\end{verbatim}
\caption{Example of HIJSON file.}
\label{fig:HIJSON}
\end{figure}

The HIJSON document is composed of different parts:

\begin{itemize}
\itemsep1pt\parskip0pt\parsep0pt
\item 
 configuration: a JSON object containing parameters and settings useful
 for the building representation. In particular three points of the local
 reference system are mapped to three couples of geographical coordinates.
 This information allows the computation of the transformation matrix used to
 translate the local coordinates to global ones.
\item
 one or more data collections: each of these lists is given in the form of
 a GeoJSON FeatureCollection, containing a number of HIJSON Elements. Since
 HIJSON Elements adhere to the GeoJSON format, each collection can be
 accepted by a GeoJSON validator. HIJSON introduces some additional rules
 that allow the adoption of this format for indoor representation. In the
 next paragraph is given a sample of HIJSON Element, with the description of
 the main differences from a standard GeoJSON Feature. 
\end{itemize}


\begin{figure}[h]
\begin{verbatim}
{
 "id": "architecture",
 "type": "FeatureCollection",
 "features": 
 [
 ...
 {
 "type": "Feature",
 "id": "room_0.1",
 "geometry": 
 {
 "type": "Polygon",
 "coordinates": 
 [ 
 [ [0, 0], [11, 0], [11, 19], [0, 19] ]
 ]
 },
 "properties": 
 {
 "class": "room",
 "parent": "level_0",
 "description": "Office of Mr. Smith",
 "tVector": [10, 20, 0],
 "rVector": [0, 0, 90]
 },
 ...
 ]
}
\end{verbatim}
\caption{Example of data collection, with the definition of an HIJSON Element.}
\label{fig:dataCollection}
\end{figure}

\subsection{HIJSON Element description}

The first additional requisite above the GeoJSON format rules is the
necessity of a unique ID, necessary for the referencing by possible
child elements. The Geometry types allowed are \texttt{Point},
\texttt{LineString} and \texttt{Polygon}. Each geometry type is used to
represent particular categories of elements (e.g.~Polygons for levels
and rooms, LineString for walls and doors, Point for furniture, etc.).
The geometry coordinates are expressed in meters, and for convention
starting at the bottom-left of the element. Unlike GeoJSON, where all
the properties are optional, in HIJSON some attributes are mandatories:

\begin{itemize}
\itemsep1pt\parskip0pt\parsep0pt
\item
 \texttt{class}: represent the element category, used to instantiate
 the appropriate semantic class;
\item
 \texttt{parent}: contains parent's id of the nodes. The reason of the
 unique id depends on this property. The HIJSON Tree is created on the
 base of parent property;
\item
 \texttt{tVector} and \texttt{rVector}: represent the translation and
 rotation relative to the parent element. The measure unit for
 translation is meter and for rotation is grades.
\end{itemize}

The definition of other properties is mandatory depending on the class
of the element: For example the classes that defines internal or
external walls require a \texttt{connections} array, containing the IDs
of the adiacent areas. This information is used by the connector
children of the element, like doors, to identify the areas linked
together. These connector elements are identified by a
boolean \texttt{connector} property.

Optional fields can be added to improve the precision of the
representation. Given the nature of the GeoJSON format from which HIJSON
derives, the elements are represented by their 2D shape, like on a
planimetry. To assign a value to the height of the object, intended as
third dimension, the property \texttt{height} can be used.

A \texttt{description} property can provide further information about
the element.

Additional optional fields can be added without restrictions, in order 
to enrich and extend the expressvity of the representation. An example of data collection is given in Figure~\ref{fig:dataCollection}.

