
\section{HIJSON structure and syntax}\label{hijson-syntax}

The HIJSON document is composed of a configuration section, followed by one or more {\tt DataCollections}, containing the actual data.

Listing~\ref{lst:hijson-example} shows a simplified HIJSON document, devoid of punctual details, to make clear to the reader the overall document structure.


\begin{lstlisting}[language=json, label={lst:hijson-example}, captionpos=b, caption=Example of HIJSON document.]
{
  "config": {
    // ...
  },
  "data": [
    ...
    {
      "id": "architecture",
      "type": "FeatureCollection",
      "features": [
        // ...
      ] 
    },
    {
      "id": "furniture_1",
      "type": "FeatureCollection",
      "features": [
        // ...
      ] 
    },
    // ...
  ]
}
\end{lstlisting}


The configuration includes parameters and settings needed for building representation in the form of a JSON Object. One of the core information in this section is defined by the correspondence between three points of the local coordinate system and three points of the real world, expressed in geographical coordinates. This is needed to ensure a seamlessly passage from local to geographical coordinate system and vice versa.

After the configuration part, the document includes a list of {\tt DataCollection}. An example
of {\tt DataCollection} is given in listing~\ref{lst:data-collection-example}.


\begin{lstlisting}[language=json, label={lst:data-collection-example}, captionpos=b,  caption=Example of {\tt DataCollection}.]
{
  "id": "architecture",
  "type": "FeatureCollection",
  "features": [
    // ...
    {
      "type": "Feature",
      "id": "room_0.1",
      "geometry": {
        "type": "Polygon",
        "coordinates": [[ [0, 0], [11, 0], [11, 19], [0, 19] ]]
      },
      "properties": {
        "class": "room",
        "parent": "level_0",
        "description": "Office of Mr. Smith",
        "tVector": [10, 20, 0],
        "rVector": [0, 0, 90]
      }
    },
    // ...
  ]
}
\end{lstlisting}

Each element of the list is given in the form of a GeoJSON {\tt
FeatureCollection}, that contains an arbitrary  number of \emph{HIJSON Elements}.
Each {\tt DataCollection} imposes a logical relationship that can be used
to group together related HIJSON Elements. Since  HIJSON Elements adhere to
the GeoJSON format, each {\tt DataCollectio}n results compliant with GeoJSON
syntax and then accepted by any GeoJSON validator. As detailed below, the
HIJSON format  introduces some additional rules that allow the adoption of
this format for indoor representation.


\subsection{HIJSON Element}

Dealing with indoor environments, there are essentially two classes of objects
that is necessary to represent. They are (a) architectural elements, like a
room, a corridor, a wall, etc. and (b) furnishings, intended in a broad sense,
such as to contain both furniture, like a desk or a chair, and/or ``smart objects'',
like an IP-cam or a thermostat.

An HIJSON Element defines a syntax to describe both geometry and properties of
an object, represents the atomic component of an HIJSON document, and is compliant with GeoJSON syntax. 
It would be a best practice to group
together related JSON Element using {\tt Data Collection}: several classification strategies
can be applied, i.e.~by grouping the elements by storey or even by room.
Alternatively, since the furnishings are more likely to change than the
architectural components of a building, these two different kinds of elements
can be isolated in different {\tt Data Collections}.

The hierarchical structure of the document gives visible form to the capability of HIJSON Elements to have children elements. A unique ID is mandatory for every HIJSON Element. 

Three Geometry types can be used here: {\tt LineString},
{\tt Point}, and {\tt Polygon}. The choice of the Geometry type to be associated to
a HIJSON Element implicitly defines the category of the element: {\tt
LineString} is used for walls and doors, {\tt Point} for furnishings, while {\tt
Polygon} may describe levels and rooms.

The Geometry coordinates are expressed in metres, by convention starting
at the bottom-left of the element, whose position is used to set-up the origin of a local coordinate frame. Unlike GeoJSON, where all properties
are optional, in HIJSON some strict requirements are imposed, and some
attributes are mandatories:

\begin{itemize}
\itemsep1pt\parskip0pt\parsep0pt
\item
 {\tt class}: represents the element category, used to instantiate
 the appropriate \emph{HIJSON Class};
\item
 {\tt parent}: contains the parent's ID of the element;
\item
 {\tt tVector}: represents the translation relative to 
 the parent element, expressed in metres;
\item
 {\tt rVector}: represents the rotation relative to 
 the parent element, expressed in nonagesimal degrees.
\end{itemize}

Specific classes may require the mandatory presence of other properties. For
example, the classes {\tt internal\_wall} and {\tt external\_wall} that
define the internal partitions and the external envelope, respectively, require a {\tt connections}
array, containing the IDs of the adiacent elements. This information is used
by the connector children of the element (e.g. doors) to identify the
areas linked together.

Given the nature of the GeoJSON format from which HIJSON derives, the elements
are represented by their 2D shape, like on a planimetry. The property {\tt
height} was introduced to assign a value to the height of the object, intended
as a third dimension.

A {\tt description} property can provide further information about
the element.
Arbitrary optional fields can be added without restrictions, in order to
enrich and extend the expressivity of the representation, or simply for the sake of 
documentation.
