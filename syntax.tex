
\section{HIJSON structure and syntax}\label{hijson-syntax}

In listing~\ref{lst:hijson-example} it is shown a simplified HIJSON document, devoid of puctual datails, to make clear to the reader the overall structure.

\begin{lstlisting}[language=json, label={lst:hijson-example}, captionpos=b, caption=Example of HIJSON document.]
{
  "config": {
    // ...
  },
  "data": [
    ...
    {
      "id": "architecture",
      "type": "FeatureCollection",
      "features": [
        // ...
      ] 
    },
    {
      "id": "furniture_1",
      "type": "FeatureCollection",
      "features": [
        // ...
      ] 
    },
    // ...
  ]
}
\end{lstlisting}


The HIJSON document is composed of a configuration section, followed by one or more {\tt DataCollections}, containing the actual data.

The configuration includes parameters and settings needed for building representation in the form of a JSON Object. One of the core information in this section is defined by the correspondence between three points of the local coordinate system and three point of the real world, expressed in geographical coordinates. This is needed to ensure a seamlessly passage from local to geographical coordinate system and vice versa.

After the configuration part, goes a {\tt DataCollection} list. Each element
of the list is given in the form of a GeoJSON {\tt FeatureCollection},
containing an arbitrary  number of {\emph HIJSON Elements}. Each {\tt
DataCollection} imposes a logical relationship that can be exploited to group
together related HIJSON Elements. Since  HIJSON Elements adhere to the GeoJSON
format, each {\tt DataCollectio}n results compliant with GeoJSON syntax and
then accepted by any GeoJSON validator.

An example of {\tt DataCollection} is given in listing~\ref{lst:data-collection-example}. HIJSON format introduces some additional rules that allow the adoption of this format for indoor representation, detailed below.

\begin{lstlisting}[language=json, label={lst:data-collection-example}, captionpos=b,  caption=Example of {\tt DataCollection}.]
{
  "id": "architecture",
  "type": "FeatureCollection",
  "features": [
    // ...
    {
      "type": "Feature",
      "id": "room_0.1",
      "geometry": {
        "type": "Polygon",
        "coordinates": [ 
          [ [0, 0], [11, 0], [11, 19], [0, 19] ]
        ]
    },
    "properties": {
      "class": "room",
      "parent": "level_0",
      "description": "Office of Mr. Smith",
      "tVector": [10, 20, 0],
      "rVector": [0, 0, 90]
    },
    // ...
  ]
}
\end{lstlisting}

\subsection{HIJSON Element}

Dealing with indoor environments, there are essentially two classes of object
that is necessary to represent. They are (a) architectural elements, like a
room, a corridor, a wall, etc. and (b) furnishings, intended in a broad sense,
such as to contain so furniture, like a desk or a chair, as ``smart objects''
like an IP-cam or a thermostat.

An HIJSON Element defines a syntax to describe both geometry and properties of
an object and represents the atomic component of an HIJSON document. It is in
turn compliant with GeoJSON syntax. It would be a best practice to group
together related JSON Element using {\tt Data Collection}: several strategies
can be applied, for example grouping by storey or even by room can be imposed.
Alternatively, since the furnishings are more likely to change than the
architectural components of a building, these two different kind of elements
can be isolated in different {\tt Data Collections}.


The hirarchical structure of the document is embodied by the possibility of the HIJSON Elements to have childern elements, so a unique ID is mandatory for each HIJSON Element. 

There are three allowed Geometry types that can be used: {\tt LineString},
{\tt Point}, and {\tt Polygon}. The choice of a Geometry type to associate to
a HIJSON Element implicitly define the category of the element: {\tt
LineString} are used for walls and doors, {\tt Point} for funishings, and {\tt
Polygon} describes levels and rooms.

The Geometry coordinates are expressed in meters, and for convention starting
at the bottom-left of the element. Unlike GeoJSON, where all the properties
are optional, in HIJSON some striclty requirements are imposed and some
attributes are mandatories:

\begin{itemize}
\itemsep1pt\parskip0pt\parsep0pt
\item
 {\tt class}: represent the element category, used to instantiate
 the appropriate {\emph HIJSON Class};
\item
 {\tt parent}: contains parent's ID of the element.;
\item
 {\tt tVector} and {\tt rVector}: represent the translation and
 rotation relative to the parent element. The measure unit for
 translation is meter and for rotation is grades.
\end{itemize}

Specific classes may require the mandatory presence of other properties. For
eexample the two classes {\tt internal_wall} and {\tt external_wall} that
define internal and external walls respectively, require a {\tt connections}
array, containing the IDs of the adiacent elements. This information is used
by the connector children of the element (e.g. like doors) to identify the
areas linked together.

Given the nature of the GeoJSON format from which HIJSON
derives, the elements are represented by their 2D shape, like on a
planimetry. To assign a value to the height of the object, intended as
third dimension, has been introduced the property {\tt height}.

A {\tt description} property can provide further information about
the element.

Arbitrary optional fields can be added without restrictions, in order to
enrich and extend the expressvity of the representation, or with the simple
documenting purpose.
