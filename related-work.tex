\section{Related work}\label{related-work}

Research on the cartographic representation of indoor environments is
extensive and heterogeneos with respect to the strategies
applied. Different information sources are used, and accuracy of the
produced solution depends on the adopted approach. In some cases the
information is obtained with automatic or semi-automatic processing of files
that describe the architectural structure of a building, such as BIM (Building Information Modeling)~\cite{Eastman:2008:BHG:1796500} and/or IFC (Industry Foundation Classes) that describe a building project \cite{6816739}. Image processing is also used to
extract topological information from floor plan images \cite{6878152}. In
other works, building informations and descriptive parameters are redefined
from scratch \cite{6418876}. Such approaches suffer from the non appropriateness of
their representive formats: images contain poor information and CAD
files are not designed for this kind of use. 

A recurring theme among the use
of carthograhic information is \emph{indoor navigation}
\cite{6878152,6418876,6816739}. The proposed approaches are very different in
this case too, and based on several strategies with some basic elements in
common. An often adopted solution is based on the representation of the
routing information as a graph, having a node for each room and an edge for
each pair of connected rooms. In some cases the edges are weighted in
function of euclidean distance. The detail level of the graph, and hence the
effective usefulness of the calculated paths, can vary depending on the
technique and the design choices applied, but in general most of the proposed
solutions retrieve information only from architectual structure. 

A subject related to navigation is the \emph{location of users}. All the considered papers
agree on the inadequacy of GNSS in indoors contexts, due to the segnificant
reduction in signal quality. To locate the exact position of a user inside a
building, the currently most applied techniques are based on fingerprinting and
triangulation of radio signals (WiFi, Bluetooth, etc.) flanked by more
original solutions based, for example, on image recognition \cite{6815564}.
User tracking issue is faced with solutions that range from the clever
utilization of inertial tracking sensors embedded in many smartphones
\cite{6815564} to the adoption of ad hoc devices \cite{6878152}.

The actual ``de-facto'' standard in terms of geospatial data representation is
the \emph{GeoJSON} format, which can be easily used for any type of geographical
annotation. In some cases it has been slightly adapted to be used in indoor
environments: it is the case of the \emph{IndoorJSON} format.

\subsection{The GeoJSON format}\label{the-geojson-format}

\emph{GeoJSON} is a geospatial data interchange format based on JSON, suitable for a
geometrical encoding of various geographic data structures. As opposed to GIS
formats, GeoJSON is an open standard. Positions need to be expressed in
geographical coordinates (usually WGS84).

GeoJSON supports some geometric primitives, including: {\tt Point},
{\tt Line\-String}, {\tt Polygon}, {\tt MultiPoint},
{\tt MultiLineString}, and {\tt MultiPolygon}. Lists of geometries are
represented by a {\tt GeometryCollection}. Geometries with additional
properties are {\tt Feature} objects. Lists of {\tt Feature} are
represented by a {\tt FeatureCollection}.

A single {\tt Feature} is composed essentially by two mandatory fields:
{\tt geometry}, which describes the object geometry accordingly to the
previously recalled primitives, and {\tt properties} which contains
additional information about the {\tt Feature}.

In GeoJSON it is possible to define complex shapes through the composition of simpler
 objects. Mainly due to its semplicity, GeoJSON is widely used and
deeply integrated in several applications and services.

\subsection{The IndoorJSON format}\label{experiences-on-indoor-json}

\emph{IndoorJSON} is a GeoJSON variant defined and used by \emph{indoor.io}, a
Finnish company devoted to indoor environment mapping. IndoorJSON is compliant
with GeoJSON syntax, and it may consist of any number of {\tt Features}
and/or {\tt FeatureCollections}. All {\tt Features} are interpreted
similarly regardless of their grouping into nested
{\tt FeatureCollections}. IndoorJSON supports all GeoJSON geometry types.

Some particular properties are used to correctly define the indoor elements:

\begin{itemize}
\itemsep1pt\parskip0pt\parsep0pt
\item
 {\tt level}: describes which level contains the feature;
\item
 {\tt geomType}: identifies the object's
 category, useful during the visualization
 process.
\end{itemize}

There are some additional but not mandatories properties, useful for
the indoor representation:

\begin{itemize}
\itemsep1pt\parskip0pt\parsep0pt
\item
 {\tt accessible}: describes if an element is walkable or not;
\item
 {\tt connector}: defines if the element is a connection between two
 levels;
\item
 {\tt direction}: describes the direction of the connection (both
 ways, only up, only down);
\end{itemize}

A syntax validator is provided by \emph{indoor.io}, but the commercial nature
of this project limits the number of tools available to deal with this
format.
