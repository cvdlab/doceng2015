\begin{abstract}

This paper introduces HIJSON\footnote{This work was partially funded with
grants by Sogei S.p.A.,the ICT company of the Italian Ministry of Economy and
Finance.}, a novel indoor cartographic document format. A software framework
is also presented, that relies on HIJSON documents and is entirely based on
web technologies. With respect to current cartographic formats, HIJSON brings
four major enhancements: (a) exposes a hierarchical structure; (b) uses local
metric coordinate systems; (c) may import external geometric models; (d)
accepts semantic extensions. The HIJSON format is designed to describe any
geometry of the \emph{indoor space} of complex buildings, capturing their
hierarchical structure, a complete representation of their topology, and all
the objects (either smart or not) contained inside. The textual representation
allows the software framework to offer a web environment in which the user is
presented with either 2D or 3D models of the indoor ambient to navigate. Such
virtually rebuilt environment, accessible via web browsers from any kind of
device, can be regarded as the platform where several applications may
coexist: IoT monitoring; realtime multi-person tracking; cross-storey user
navigation, through an algorithm that automatically finds valid walkable
routes, taking into account both architectural obstacles and furniture. The
semantic extensions supported by the HIJSON framework architecture encapsulate
the details about communication protocols, rendering style, and exchanged and
displayed information, allowing the HIJSON format to be extended with any sort
of models of objects, sensors or behaviors.

\end{abstract}
