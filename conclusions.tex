\section{Conclusions}\label{conclusions}

In this paper a novel format for indoor cartographical description has been
introduced, named HIJSON. Utilization of local metric coordinate system,
avoiding the manipulation of geographical coordinate really inconvenient when
dealing with indoor spaces and objects, greatly simplify the drawing up
process of the document. Process that can be further improved realizing a
graphical editor to assist the user during the description of the indoor
space. The implementation of such an editor is already programmed.

The HIJSON format focuses on a hierarchical representation of the indoor
spaces that allows for completely capturing their topology. On the basis of
this representation a virtual web environment can be rebuilt working as a
unifying platform to run a bunch of different applications. The reference
architecture of such a platform has been also implemented and described in
this work. The architecture supports a range of applications: IoT monitoring,
realtime-multiperson tracking and user cross-storey navigation are already
implemented and described. A very convenient way to extend representation
capabilities of smart objects is also mentioned as semantic extensions. These
extensions (which affects both document format and web framework) might be
easly collected in a public repository. Community could both use public
available extensions or contribute by mapping new (smart) objects inside the
HIJSON document.
