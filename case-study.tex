\section{Case study}\label{case-study}

As the first case study has been taken into account the need of Sogei S.p.A. to support its data center maintaince, which is subject to strict access control policies. 
Lo scenario considerato consiste nell'intervanto di manutenzione da parte di un operatore che deve muoversi nell'ambiente e localizzare all'interno di un datacenter dalle enormi dimensioni la macchina su cui operare, individuandolo tra migliaia di rack di aspetto simile. Tale operatore sarà quindi equipaggiato con un client explorer. Il client explorer guiderà l'operatore fino al sottosistema su cui intervenire, notificandone al contempo la posizione ad ogni istante.

Le figure responsabili, attraverso il supervisor client possono monitorare la situazione degli smart object, che in questo caso spaziano dalle webcam ai sistemi di allarme antincendio ma comprendono anche le macchine fische, sicchè sia possibile monitorarne lo stato di funzionamento, temperatura di esercizio e carico di lavoro...


The real­time awareness of the relative positions between
the ‘maintenance man' andthe rack ­ containing the machine­ will help us
increasing safety and reducing intervention times.
Infatti, le figure responsabili, con diritti di accesso al client supervisor, avranno quindi modo di monitorare le posizione degli operatori attivi all'interno del datacenter verificando che essi non devino su percorsi non autorizzati.

In particolare tali supervisori, nello scenario di manutenzione in analisi, tracciando l'intervento dell'operatore in tempo reale possono scaricare il sistema target della manutenzione da tutti i servizi che offre, migrando le corrsipondenti macchine virtuali e distribuendole su altri sistemi, garantendo così continuità di servizio e diminuendo il global risk factor. Terminato l'intervento tecnico, lo stato precedente può essere ripristinato. 




%----------------------------------------------------------------------------

% right sub­system (inside the rightrack among thousands). We will call them the
% ‘maintenance man’. The real­time awareness of the relative positions between
% the ‘maintenance man' andthe rack ­ containing the sub­system ­ will help us
% reducing intervention times and increasing safety. If you knew when the
% maintenance process starts, you couldautomatically move,in real time,
% services, that is virtual machines, to other systems,thus
% maintainingcontinuity of services and, in the same time, reducing the
% globalrisk factor. Whenmaintenance is over, and the technician moves away, you
% couldinstantly restore thepre­existing conditions of services, that is,
% immediately after havingperformed an outright test.

% We are now trying to
% verify the possibility to perform a realtime control over the maintenance
% workflow in a complex Data Center. 

% Inparticular, wehave to support the process of those in charge of carrying out
% the ticket­maintenance reaching, as quick as possible and without error, the
% right sub­system (inside the rightrack among thousands). We will call them the
% ‘maintenance man’. The real­time awareness of the relative positions between
% the ‘maintenance man' andthe rack ­ containing the sub­system ­ will help us
% reducing intervention times and increasing safety. If you knew when the
% maintenance process starts, you couldautomatically move,in real time,
% services, that is virtual machines, to other systems,thus
% maintainingcontinuity of services and, in the same time, reducing the
% globalrisk factor. Whenmaintenance is over, and the technician moves away, you
% couldinstantly restore thepre­existing conditions of services, that is,
% immediately after havingperformed anoutright test.

