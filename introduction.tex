\section{Introduction}\label{introduction}


Among the many activities of R\&D Sogei, we are now analyzing the
computingofposition data during the transition between outdoor and
indoorscenarios. In orderto achieve our purpose we are experimenting
severaltechnologies, such asGNSS (Global Navigation Satellite system), WI­FI,
LTE(Long Term Evolution),SDR(Software Defined Radio). 






We are now trying to
verify the possibility to perform a realtime control over the maintenance
workflow in a complex Data Center. 


Inparticular, wehave to support the process of those in charge of carrying out the ticket­maintenance reaching, as quick as possible and without error, the right sub­system
(inside the rightrack among thousands). We will call them the ‘maintenance man’. The real­time awareness of the relative positions between the ‘maintenance man' andthe rack ­ containing
the sub­system ­ will help us reducing intervention times and increasing safety.
If you knew when the maintenance process starts, you couldautomatically
move,in real time, services, that is virtual machines, to other systems,thus
maintainingcontinuity of services and, in the same time, reducing the
globalrisk factor. Whenmaintenance is over, and the technician moves away, you
couldinstantly restore thepre­existing conditions of services, that is,
immediately after havingperformed anoutright test.



VASTI AMBIENTI 



MOLTI SENSORI E SISTEMI DI RILEVAZIONE DELLA POSIZIONE DIFFERENTI

CONTINUOUS OUTDOOR-INDOOR NAVIGATION


MONITORAGGIO DI SCENARI UNIFICATI DI INTERVENTO UOMO MACCHINA


PERSONALE TECNICO DEVE ESSERE GUIDATO 



IL PASSAGGIO DA SERVER A IOT È IMMEDIATO


MONITORAGGIO UNIFICATO DEGLI SMART OBJECTS IOT






SERVER REAL MACHINE (that can be assimialted to smart objects in the IoT context, since )






DESCRIZIONE DELLE PROBLEMATICHE 

TALI PROBLEMATICHE SONO TIPICHE DI SCENARI IN CUI SIA NECESSARIO IL CONTROLLO DEGLI ACCESSI
LA SUPERVISIONE, ...






IN QUESTO SCENARIO SI PONE IL LAVORO DESCRITTO IN QUESTO PAPER, 
IN CUI LE NECESSIT`A DESCRITTE VENGONO AFFRONTATE PARTENDO DALLE BASI,
OVVERO DEFINENDO UN FORMATO DI DOCUMENTO PER LA DESCRIZIONE ASTRATTA DI AMBIENTI INDOOR.
DESCRITTO L'AMBIENTE DEVE QUINDI ESSERE POSSIBILE RICOSTRUIRE VIRTUALMENTE L'AMBIENTE
RENDERLO LARGAMENTE ACCESSIBILE VIA WEB, E MONTARE SU QUESTA RAPPRESENTAZIONE VIRTUALE 
LA POSSIBILIT`A DI INTERAGIRE CON GLI OGGETTI ALL'INTERNO DELL'AMBIENTE. L'INTERAZIONE DEVE CONSISTERE DA UN LATO NALLA POSSIBILIT`A DI RICEVERE INFORMAZIONI DALL'OGGETTO, MA ANCHE DI INVIARE COMANDI ALL'OGGETTP.

SI REALIZZA IN QUESTO MODO UNO SCENARIO IN CUI UN SUPERVISORE INTERAGISCE CON L'AMBIENTE IN CUI SI MUOVE UN EXPLORER (O MANUTENTORE) POTENDO IL SUPERVISORE AVERE IMMEDIATA NOTIFICA DELLA POSIZIONE DEL MANUTENTORE, E AVENDO UN QUADRO COMPLETO FORNITO DAGLI OGGETTI SMART PRESENTI NELL'AMBIENTE REALE ASSIEME ALL'EXPLORER.

D'ALTRA PARTE PER GRANDI SPAZI ANCHE L'EXPLORER PUO ESSERE SUPPORTATO DAL SISTEMA CHE AVENDO COMPLETA CONOSCENZA DELLA TOPOLOGIA E DELLA GEOMETRIA DELL'AMBIENTE, NONCHE DEGLI OGGETTI IN ESSO CONTENUTI, POSSIEDE TUTTE LE INFORMAZIONI NECESSARIE PER GUIDARE L'EXPLORER ATTRAVERSO L'AMBIENTE.

NULLA IMPEDISCE DI MIXARE LE NECESSITÀ DI EXPLORER E SUPERVISOR, REALIZZANDO UN EXPLORER CHE NAVIGA NELL'AMBIENTE REALE ED IN ESSO RICEVE INFORMAZIONE E A CONTEMPO PUÒ INVIARE COMANDI AGLI SMART OBJECT INTORNO A SE.

\ldots{}

The remainder of this document is organized as follows. In Section II is
provided an overview of the state of the art in the field of indoor document
standard and related applications. Section III is devoted to describe the
advances introduced by the novel cartographic document proposed, while section
IV presents the document syntax. Section V reports about the toolkit
specifically developed to handle the new document format. In Section VI is
depicted the overall architecture and the implementation of the web based
application framework, which is in turn used to achieve the objectives stated above.
Finally, Section VII proposes some conclusive remarks and future developments.
