\section{Introduction}\label{introduction}

An indoor environment interactive mapping consists of a virtual reconstruction of a real indoor space in which the user can move around and then interact with the objects which are found the same position they actually occupy in the real world.
So an indoor interactive mapping, can be thought of a specialized and very evoluted user interface capable of giving a glimpse of a section of the real world that the user can handle in a natural and intuitive way. 

Such a reconstructed virtual indoor environment is very interesting as it can be considered a general platform where many different applications can rely upon.

Both promising and already well explored researching fileds find in virtual indoor mapping the perfect context to be integrated into.

In particular, for ambineces which massive presence of sensor-equipped (or ``smart'') objects, which realize the so-called IoT, the indoor interactive mapping represents an ideal integrated interface for IoT monitoring system.

It can be the container of indoor navigation systems, giving to the user that need to be routed across an indoor environment the possibility to interact with objects along the suggested paths.

Futhermore, in conjunction with the advancements in the fields of user indoor location, achieved via a variety of positioning systems like GNSS (Global Navigation Satellite system), Wi-Fi, LTE (Long Term Evolution), SDR(Software Defined Radio), it represents the most natural interface to perform realtime access monitoring and multiperson tracking.

To enable such a interactive mapping platform, it is indispensable a descriptive representation of the indoor environment. This means entering in the field of indoor cartography, which as digital evolution of plain floor plans, is arrived to arouse the interest of big players like Google that has integrated into Google Maps indoor plans of specific locations of interest \cite{indoormaps}. In general, can be considered ``of interest'' (such to justify and motivate indoor cartographic applications) public or commercial places of vast dimensions, as for example airports, train stations, shopping mall, or alterantively private buildings subject to strict access control, like warehouse, logistics centers, datacenter. 

Despite of the growing attention around indoor cartography, efforts to bring out definitions of open formats for indoor representation are few and partial, certainly not inteded to support interactive indoor mapping which is instead the main porpuse this paper. 

The work, jointly developed by {\bf SoGeI} ({\bf So}ciet\`a {\bf Ge}nerale d'{\bf I}nformatica), a company fully owned by Italian  Ministry of Economy and Finance, and the {\bf CVDLAB} (Computational Visual Design Laboratory) of the Computer Science and Automation Department of ``Roma Tre'' University, is inspired by the necessities of SoGeI itself, which runs one of the largest data center of Italy, then requiring strict access control policies which have to be composed with indispensable human/machine coordinate maintenance scenarios. Support for these scenarios, where real time awarness of the maintainer position inside the data center helps to both reduce intervention and increase safety, has been adopted as the case study of the interactive indoor mapping based on the indoor cartographical format proposed.


The remainder of this document is organized as follows. In Section~\ref{introduction} we
provide an overview of the state of the art in the field of indoor document
standards and related applications. Section~\ref{advances} is devoted to describe the
advances introduced by the novel cartographic document proposed, while section~\ref{hijson-syntax} presents the document syntax. Section~\ref{hijson-toolkit} reports about the toolkit
specifically developed to handle the new document format. In Section~\ref{hjson-web-framework} it is
depicted the overall architecture and the implementation of the web based
application framework, which is in turn used to achieve the objectives stated above. Section~\ref{use-case} presents a case-study application of the document format discussed in this paper.
Finally, Section~\ref{conclusions} proposes some conclusive remarks and future developments.

% -----------------------------------------------------------------------------


% VASTI AMBIENTI 
% MOLTI SENSORI E SISTEMI DI RILEVAZIONE DELLA POSIZIONE DIFFERENTI
% CONTINUOUS OUTDOOR-INDOOR NAVIGATION
% MONITORAGGIO DI SCENARI UNIFICATI DI INTERVENTO UOMO-MACCHINA
% PERSONALE TECNICO DEVE ESSERE GUIDATO 
% IL PASSAGGIO DA SERVER A IOT È IMMEDIATO
% MONITORAGGIO UNIFICATO DEGLI SMART OBJECTS IOT
% un ambiente indoor virtuale è l'ambiente perfetto per andare a monitorare l'IoT.

% IN QUESTO SCENARIO SI PONE IL LAVORO DESCRITTO IN QUESTO PAPER, 
% IN CUI LE NECESSIT`A DESCRITTE VENGONO AFFRONTATE PARTENDO DALLE BASI,
% OVVERO DEFINENDO UN FORMATO DI DOCUMENTO PER LA DESCRIZIONE ASTRATTA DI AMBIENTI INDOOR.
% DESCRITTO L'AMBIENTE DEVE QUINDI ESSERE POSSIBILE RICOSTRUIRE VIRTUALMENTE L'AMBIENTE
% RENDERLO LARGAMENTE ACCESSIBILE VIA WEB, E MONTARE SU QUESTA RAPPRESENTAZIONE VIRTUALE 
% LA POSSIBILIT`A DI INTERAGIRE CON GLI OGGETTI ALL'INTERNO DELL'AMBIENTE. L'INTERAZIONE DEVE CONSISTERE DA UN LATO NALLA POSSIBILIT`A DI RICEVERE INFORMAZIONI DALL'OGGETTO, MA ANCHE DI INVIARE COMANDI ALL'OGGETTP.

% SI REALIZZA IN QUESTO MODO UNO SCENARIO IN CUI UN SUPERVISORE INTERAGISCE CON L'AMBIENTE IN CUI SI MUOVE UN EXPLORER (O MANUTENTORE) POTENDO IL SUPERVISORE AVERE IMMEDIATA NOTIFICA DELLA POSIZIONE DEL MANUTENTORE, E AVENDO UN QUADRO COMPLETO FORNITO DAGLI OGGETTI SMART PRESENTI NELL'AMBIENTE REALE ASSIEME ALL'EXPLORER.

% D'ALTRA PARTE PER GRANDI SPAZI ANCHE L'EXPLORER PUO ESSERE SUPPORTATO DAL SISTEMA CHE AVENDO COMPLETA CONOSCENZA DELLA TOPOLOGIA E DELLA GEOMETRIA DELL'AMBIENTE, NONCHE DEGLI OGGETTI IN ESSO CONTENUTI, POSSIEDE TUTTE LE INFORMAZIONI NECESSARIE PER GUIDARE L'EXPLORER ATTRAVERSO L'AMBIENTE.

% NULLA IMPEDISCE DI MIXARE LE NECESSITÀ DI EXPLORER E SUPERVISOR, REALIZZANDO UN EXPLORER CHE NAVIGA NELL'AMBIENTE REALE ED IN ESSO RICEVE INFORMAZIONE E A CONTEMPO PUÒ INVIARE COMANDI AGLI SMART OBJECT INTORNO A SE.
